\documentclass{article} % For LaTeX2e
\usepackage{cos424,times}
\usepackage{url}
\usepackage{graphicx}
\usepackage{hyperref}
\usepackage{amsmath, bbold}
\usepackage{bm}
\usepackage{biblatex}
\bibliography{bib.bib}

\newcommand{\e}{\text{e}}
\newcommand{\bt}{\mathbf}

\title{Kaggle Competition: Flavors of Physics}


\author{
Kenan Farmer\\
Computer Science\\
\texttt{kfarmer@} \\
\And
Charles Stahl \\
Physics \\
\texttt{cnstahl@} \\
\And
Meir Hirsch\\
Computer Science \\
\texttt{ehirsch@} \\
}

\newcommand{\fix}{\marginpar{FIX}}
\newcommand{\new}{\marginpar{NEW}}

\begin{document}

\maketitle

\subsection*{Introduction}

The Large Hadron Collider (LHC) is a particle accelerator in Europe. It is designed to test high energy physics, specifically the standard model. The standard model is our current understanding of how the universe behaves. 

One prediction of the standard model is conservation of lepton number \cite{griff87}. Lepton number is the total number of leptons minus the total number of antileptons,
\begin{align}
L=n_{\ell }-n_{\overline {\ell }}~.
\end{align}
There are two types of leptons: charged leptons such as electrons muons and tauons, and neutral leptons, which are called neutrinos. They can be further divided into families. Electrons and electron neutrinos are in the electron family, etc.

Neutrinos have a very small mass, six to nine orders of magnitude smaller than the electron mass. If their mass were exactly zero, Lepton number would be conserved for each family individually. However, the minuscule neutrino mass allows some lepton family number violations. 

The $\tau\to\mu\mu\mu$ reaction is one such violation. The Feynman diagram for the decay is in figure~\ref{fig:feyn}.

\begin{figure}[hb]
	\centering
	\includegraphics[width=.5\textwidth]{front_page}
	\caption{\textbf{Decay of $\tau$ particle to three $\mu$ particles.} Individual lepton family numbers are not conserved, as the initial tau number is either 1 or -1. However, the total lepton number is conserved because the top two muons cancel each other.}
	\label{fig:feyn}
\end{figure}

Detecting or not detecting this decay then helps put bounds on the neutrino mass, a current goal of particle physics \cite{lhcb15}.

\subsection*{Data Description}

This will be a classification problem using all real-valued data.

Majority of the data variables are the momenta of the leptons. Many of the variables are highly correlated to each other due to their similarity in function (i.e. isolationa, isolationb, isolationc ... are all different columns of the "track isolation variable" \cite{kaggleComp}). As for size of the data is consistent of 67553 rows and 51 columns, the testing data is consistent of 855819 and 47 columns. 

Since the decay was not known to exist at the time of the challenge, there are no recorded signal events. The decay can only happen within a small mass window, though, so any events outside that window must be background. For this reason, the background events are real data. Signal events are simulated though. 

This leads to the possibility of using the mass to predict signal or background, or use some method that relies on being able to distinguish real from simulated data. The correlation and agreement tests prevent this.

\subsection*{Methodologies to Explore}
We will use a suite of classification algorithms with various parameters to classify if the lepton decay has occurred or has not occurred. We will use sci-kit learn for this signal detection among noise application. Our goal is  to classify the data into two classes that can be interpreted as the occurrence of an event. Though we are planning on using a variety of classification algorithms there are two tests given by the competition makers to ensure validity of each methodology. We will use the following tests to guide our choice of classification models and selection of  parameters.

We will test each Methodology with the following tests to ensure precise results in accordance with the competition requirements. The competition has both an Agreement Test and a Correlation test to improve the quality of methodologies.

The Correlation Test is to ensure that the Methodology does not have any correlation with the mass of the particle being inspected. This correlation could lead to false signaling and obfuscate our classification algorithm performance. Though mass is not currently included as a column or observation variable, it is used to run the Cramer-von Mises(cvm) test and so we will use the competition's Correlation test to evaluate if any significant correlation with latent mass of the particle exists.

The Agreement test is to ensure that each classifier does not have a large discrepancy between simulated data and real data. Since the classifier is trained on a mix of simulated signal and real data background, it is possible to reach a high performance by picking features that are not perfectly modeled in the simulation. The competition requires that the classifier not have a large discrepancy when applied to real and simulated data. \cite{kaggleComp}

\subsection*{Metrics}
We will follow the guide line of the competition admins for measuring our analysis performance. The main metric we will use to evaluate our methodologies is the Weighted Area under the ROC curve. The ROC curve is divided into sections based on the True Positive Rate (TPR). To calculate the total area, multiply the area with TPR in [0., 0.2] by weight 2.0, the area with TPR in [0.2, 0.4] by 1.5, the area with TPR [0.4, 0.6] with weight 1.0, and the area with TPR [0.6, 0.8] with weight 0.5. Anything above a TPR of 0.8 has weight 0. This metric is illustrated in the following figure:
\begin{center}
	\includegraphics[scale = .3]{roc_optimistic}
\end{center} 
These weights were chosen to match the evaluation methodology used by CERN scientists.\cite{kaggleComp}.

\printbibliography

\end{document}
