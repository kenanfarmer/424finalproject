\documentclass{article} % For LaTeX2e
\usepackage{cos424,times}
\usepackage{url}
\usepackage{graphicx}
\usepackage{hyperref}
\usepackage{amsmath}

\usepackage{biblatex}
\bibliography{bib.bib}

\title{Already Asked: Detecting Duplicate Question Pairs in the Quora Dataset}


\author{
Meir Hirsch \\
Computer Science\\
\texttt{ehirsch@} \\
\And
Charles Stahl \\
Physics \\
\texttt{cnstahl@} \\
\And
Kenan Farmer\\
Computer Science \\
\texttt{kfarmer@}\\
}

\newcommand{\fix}{\marginpar{FIX}}
\newcommand{\new}{\marginpar{NEW}}
\newcommand{\wordtvec}{\texttt{word2vec}}

\begin{document}

\maketitle

\begin{abstract}
This project uses a dataset posted on Kaggle~\cite{kaggleComp} to detect duplicate question pairs on Quora, a popular question-and-answer website~\cite{quora} that received 100 million monthly unique visitors as of March 2016~\cite{qvisit}. The training data includes 404,290 question pairs chosen from 537,933 questions, classified as duplicates or not. The testing set was 2.3 million unlabeled pairs, with 4.2 million unique questions. The methods we used consisted of counting the overlapping words, bag-of-words (BOW) methods, \wordtvec, and word nets to reduce the dimensionality of the data. We then used multinomial na\"ive Bayes, \_\_stochastic gradient descent\_\_, logistic regression, and random forest models to classify duplicates. The best method was BOW with logistic regression.
\end{abstract}

\section{Introduction}

Quora is a website that strives to connect people from different backgrounds to be able to answer questions whose answers are ``either locked in people’s heads, or only accessible to select groups" \cite{quora}. Questions range from ``What is the most embarrassing text message you have sent to the wrong person?" to ``How do I identify entities in natural language search query?" to ``Do virtual particles and energy in vacuum really exist?"

Once a question is answered, any other person should be able to find the response, and the question never needs to be asked again. However, sometimes people are unable to find the questions they want and end up asking the same question again. This counteracts Quora's vision of having ``only one version of each question\dots [not] a left wing version, a right wing version, a western version, and an eastern version"~\cite{quora}. Duplicate questions also waste resources for the website, the answerers, and future viewers who will either see one answer or the other, but not the canonical response the website has to offer.

Quora currently identifies duplicate questions with a Random Forest model~\cite{kaggleComp}. Quora does not have a method for users to mark questions as duplicates, although they allow users to redirect questions. Stack Exchange, a similar question-answer website, does allow users to mark questions as duplicates~\cite{stackdup}. 

To improve their duplicate identification, Quora released a dataset of question pairs on Kaggle, a platform for data science-based prediction and classification problems~\cite{kaggleComp}. In this project, we identify features and statistical machine learning methods that effectively detect duplicates in the Quora questions. This paper first explores related duplicate detection projects and explains the data used. It then describes the features we chose to extract from the data, with a special emphasis on \wordtvec, a method for word embedding. We then present results, along with analysis and conclusions drawn. 



\section{Related Work}

Zhang, et. al. created a duplicate detection tool for use on the Stack Overflow (SO), a website in the Stack Exchange network~\cite{Zhang2015}. It uses the title, content, and tags of the questions to assign topics using Latent Dirichlet Allocation (LDA)~\cite{Blei03}. The method then predicts duplicates by comparing topic distributions.

Since Quora questions include only the single sentence question, they are nearly equivalent to just the title of an SO question. The tags and content on SO are more significant and better indicate the meaning of a question, meaning this method is not useful on the Quora dataset. This analysis also only used bag-of-words text similarity and LDA to detect duplicates.

An earlier paper focused on detecting duplicate bug reports~\cite{Runeson2007}. The motivation of this problem is similar, centering on the wasted time used to identify and respond to duplicate reports. This study used standard NLP tools along with a term-frequency vector space to define distance measured between reports. 



\section{The Data}

Each line of the training set consists of a pair ID, two question IDs, the text of two questions, and an indication of whether the questions have the same intent. The testing set only contains a pair ID and the text of two questions. The questions are usually between 7 and 12 words in length. As mentioned in the abstract, the testing set is significantly larger than the training set. The goal of the competition is to submit a file with pair IDs from the testing set along with a probability that that pair is a duplicate. It then reports the log loss calculated on 35\% of the set, reserving the other 65\% for the final ranking.

We decided to approach the problem differently than the method in~\cite{Zhang2015}. While they chose to model each question and then look for duplicates, we decided that method would not be as helpful. Since we are looking for duplicates rather than related questions, the all-to-all interactions would be very sparse. Therefore we chose to model the problem as binary classification on question pairs, extracting features from each pair. 

As in all machine learning projects, we need a metric to assess the quality of our model. We decided to use the competition's metric, the log loss of the probabilities of duplicates. This is defined as the negative sum of the probabilities we assign to the correct class, or
\begin{align}
\text{log loss} = -\frac{1}{N}\sum_n \left(y_n\log p_n + (1-y_n)\log(1-p_n)\right),
\end{align} 
where $y_n$ is the true class label and $p_n$ is the probability we assign to the pair of being a duplicate. 

We did however impose some restrictions on what information we would use. The competition exposed the text of the testing data but not the labels. Since the text of the testing data would be available in a real-world application of this problem, we decided that training models on this text would not be considered double-dipping, especially since we did not have labels for this text. 

It is possible, however, to extract the proportion of positive samples in the testing data through the reported log loss. If a submission consists of only a constant probability $p$, the log loss is
\begin{align}
\text{log loss} &= -\frac{1}{N}\sum_n \left(y_n\log p + (1-y_n)\log(1-p)\right)
	\nonumber\\
&= -\left(y'\log p + (1-y')\log(1-p)\right),
\end{align}
or 
\begin{align}
y' = \frac{\text{log loss} + \log(1-p)}{\log(1-p)-\log p},
\end{align}
where $y'$ is the proportion of positive test samples. A constant test prediction of 0.75 results in a log loss of 1.19485, meaning the proportion of positive test samples is 0.17426. Simply guessing this probability results in a log loss of 0.46258. 

We could in principle use this data to structure our submissions, so that they have a mean of 0.17426. However, since this data would not be accessible in the real world, we decided to not use it in our methods. Furthermore, the proportion in the set used for final ranking may be different that in the 35\% used for initial ranking.

We also chose not use use neural networks for prediction, despite the fact that most people at the top of the leaderboard probably did. This allowed us to have more easily interpretable results, but led to less-than-stellar submissions. 

\section{Methods}

The training data was read from csv files using Pandas~\cite{pandas}. Lemmatization and stop word removal was completed using the Spacy package, a package for ``Industrial-Strength Natural Language Processing"~\cite{spacy}. Another NLP tool used is the Natural Language Toolkit~\cite{nltk}.

\subsection{Feature Engineering} \label{sub:features}

Our na\"ive method was to count the number of words in common between the two questions in the pair. A slight variant on this was listed on Kaggle as a benchmark. We improved upon this by counting how many words are in common and how many are distinct, and then by separately counting the number of nouns, verbs, named entities and \_\_\_\_ in common between questions. We also included a count of how many of each part of speech occurs in one sentence but not the other. 

Our next was a BOW method, but instead of defining an individual bag for each question, we defined a bag for each pair of questions, only including words unique to one question or the other. We then defined a second bag, only including distinct words. The second bag did not treat words unique to sentence $a$ differently from those unique to sentence $b$.

A promising source of features was word embeddings, using Gensim's implementation of \wordtvec~\cite{gensim} and Spacy's built-in \wordtvec\ embedding. Word embeddings were introduced in the early 2000's to address the high rate of orthogonality inherent in BOW-based methods~\cite{Bengio03}. Since the BOW methods only look at counts of words, the meanings of the words are not encoded. This leads to high-dimensional, very sparse vectors, leading to the high rate of orthogonality.

A canonical example is from~\cite{kusner15}. Consider the two sentences
\begin{center}
\text{Obama speaks to the media in Illinois.}
\end{center}
and 
\begin{center}
\text{The President greets the press in Chicago}.
\end{center}
These sentences are orthogonal in a space that assigns each word its own dimension. Therefore they have a cosine similarity of 0. However, an effective similarity measure should indicate that these sentences are very closely related.

Word embedding looks at a large corpus to find how different words are related and connected. It creates a space with dimension lower than the number of words, so that each word is represented as a non-sparse vector. Words with similar meaning are closer in this space, giving a high similarity score for the above sentences. Furthermore, algebraic manipulation of these vectors carries significance. Ideally this would lead to usable analogies, such as \textit{Paris - France + Italy = Rome}, or \textit{suhi - Japan + Germany = bratwurst}.

The 2013 implementation from Google, called \wordtvec, uses a shallow neural network to embed words in the vector space~\cite{word2vec}. For a detailed description of \wordtvec\ see subsection~\ref{sub:detail}.

From the \wordtvec\ embeddings of the words in each question, we can compute the word mover's distance (WMD)~\cite{kusner15}. For sentences $M$ and $N$ of length $J$ and $K$ with words $m^j$ and $n^k$, the WMD is defined as 
\begin{align}
\text{WMD}(M,N) = \sum_{j=1}^J\min_{n^k\in N}\left|m^j-n^k\right|,
\end{align}
the total distance needed to move each word in $M$ to the nearest word in $N$.

The WordNet methods were implemented using NLTK. WordNet is a database of English words and meanings created manually at Princeton~\cite{wordnet}. Words are arranged in syntactic sets based on parts of speech, and connected to their synonyms and antonyms. It is possible to find similarity measures between words using these groups of words. 

For our implementation, we \_\_\_\_

\subsection{One method in detail: \wordtvec} \label{sub:detail}

The \wordtvec\ method introduced in subsection~\ref{sub:features} merits further discussion. Training the model uses a 2-layer neural network. The implementation we used was hidden, but understanding the working is still useful. The presentation here is based on tutorials from~\cite{tensorflow} and~\cite{mccormick}.

As previously mentioned, the motivation of word embeddings is to find a vector representation for words that is not as sparse as count-based methods. To do this, we first train a neural network to learn the relationships between words and environments, or histories. The Continuous Bag-of-Words model predicts words based on environments while the skip-gram model predicts environments based on individual words. This uses a 2-layer neural network, as seen in figure~\ref{fig:skipgram}. 

\begin{figure}[h]
	\centering
	\includegraphics[width=.5\textwidth]{skip_gram_net_arch}
	\caption{\textbf{Architecture for Skip-Gram Model.} The network contains two layers after the input vector. The output layer predicts the word, but the hidden layer contains the weights used for embedding. Image from~\cite{mccormick}.}
	\label{fig:skipgram}
\end{figure}

For a vocabulary of size $N$, the input is a $N$-component vector one-hot encoded for the input word $v$. The output vector also has $N$ components, each of which is the probability that a word in the environment of $v$ is the word corresponding to that component. 

Neural networks work using neurons, each of which has an input and weights. In the skip-gram architecture, each neuron in the hidden layer has the input vector as its input. Each neuron in the output layer has all hidden neurons as input. Each training step consists of choosing a word to train on, feeding it though the network, and calculating the loss of the output vector. The parameters $\theta$ of the model are then moved slightly in the direction of the gradient of the loss w.r.t. $\theta$.

Once the model is trained, the word embeddings can be extracted. Consider a model with $H$ hidden neurons. Since each is linear, the vector of activations $h$ is just
\begin{align}
h=Mv, \label{eqn:hidden}
\end{align}
where $M$ is a $H$ by $N$ matrix containing the weights of all neurons. Since the words are one-hot encoded in $v$, the output of the hidden layer will be a single column of $M$. 

The output of the network is a softmax layer defined such that each component is between 0 and 1, and the components sum to 1. These are interpreted as the probability that a word will appear in the environment of the input word.

The key to \wordtvec\ is in looking at the output of the hidden layer. In fact, once the model is trained the output layer can be removed. Consider two words whose output vectors are similar. This means the two words are likely to occur in similar environments. This can be interpreted as the two words being semantically related. 

The last step, to actually obtain dense representations of the words in a vector space, is to realize that since the output vector is a linear transformation of the output of the hidden layer, two words with similar weights in the hidden layer will be semantically related. As described after equation~\ref{eqn:hidden}, these outputs are just the columns of $M$, which can be called the word vectors. 

An impressive display of the significance of \wordtvec\ is the capability to form analogies through algebraic manipulation of vectors. For example, starting with the vector for ``Paris," one can subtract the vector for ``France" and add the vector for ``Italy." The nearest vector in the space will then be ``Rome." A list of such analogies is in table~\ref{tab:anal}, including the imperfect analogies.

\begin{table} [tb]
\centerline{
\begin{tabular}{|c||c|c|c|}
	\hline
	Relationship            &  Example 1             &  Example 2             &   Example 3           \\
	\hline
	France - Paris          &   Italy: Rome & Japan: Tokyo  & Florida: Tallahassee    \\
	big - bigger            & small: larger & cold: colder  & quick: quicker \\
	Miami - Florida         & Baltimore: Maryland  & Dallas: Texas     & Kona: Hawaii    \\
	Einstein - scientist    & Messi: midfielder & Mozart: violinist & Picasso: painter \\
	Sarkozy - France        & Berlusconi: Italy & Merkel: Germany   & Koizumi: Japan  \\
	copper - Cu             & zinc: Zn          & gold: Au          & uranium: plutonium    \\
	Berlusconi - Silvio     & Sarkozy: Nicolas  & Putin: Medvedev   & Obama: Barack   \\
	Microsoft - Windows     & Google: Android   & IBM: Linux        & Apple: iPhone   \\
	Microsoft - Ballmer     & Google: Yahoo     & IBM: McNealy      & Apple: Jobs     \\
	Japan - sushi           & Germany: bratwurst & France: tapas    & USA: pizza      \\
	\hline
\end{tabular}}
\caption{\textbf{Analogy examples from \wordtvec}, taken from the initial \wordtvec\ paper,~\cite{word2vec}. Note the analogies are not perfect, but generally good.}
\label{tab:anal}
\end{table}

\section{Results}

The results are summarized in table~\ref{tab:res}.

\begin{table}[h]
\centerline{
\begin{tabular}{|c||c|c|c|c|}
	\hline
	Log loss       &  MNB    &  \_\_SGD\_\_    & LogReg  & RandForest \\
	\hline
	Word Count     & 0.96641 & 0.55215 & 0.54342 & 0.49019 \\
	Bag of Words   & 0.51773 & 0.46103 & 0.41490 & 0.61665 \\
	BOW, tf-idf    & 0.45464 & 0.52297 & 0.42771 & 0.60837 \\
	WordCount      &  &  &  &  \\
	\wordtvec      &  &  &  &  \\
	Ensemble       &  &  &  &  \\
	\hline
	\end{tabular}}
	\caption{\textbf{Log loss results for various models}}
	\label{tab:res}
\end{table}

\_\_\_\_ performed the best, blah blah blah

\subsection{Word Counts} \label{sub:count_res}

\begin{figure}[]
	\centering
	\includegraphics[width=.49\textwidth]{roc}
	\includegraphics[width=.49\textwidth]{roc_wc}
	\caption{\textbf{ROC curves for word count and BOW methods.}}
	\label{roc_wc}
\end{figure}

\subsection{Bag-of-Words} \label{sub:bow_res}

\subsection{WordNet} \label{sub:wnet_res}

\subsection{\wordtvec}
The word to Vector implementation results were extremely puzzling. We can break down our feature set that was input into major categories.

Once each word in a sentence is vectorized there are many ways to filter and evaluate the two collection of vectors. One implementation is known as Word mover's distance. Let a sentence be a list of words. In our case we have lemmatized each word in a sentence and removed stop words using NLTK's stop word corpus. Word movers distance is defined as taking each word inspected in one sentence and finding the minimum distance to a vector in sentence 2. Using this min distance, we sum the total for each vector in sentence1. We sum this with the process repeated for sentence2 words compared to sentence 1 words. We may write $x_1$ as a word in sentence one and $v_1 = \{x_1,...\}$ the collection of words in sentence 1 and $x_2, v_2$ defined analogously for sentence 2. This allows us to mathematically define the word mover's distance as:
$$ \sum_{x_1 \in v_1} \min_j{dist(x_1,x_{2j}) } + \sum_{x_2 \in v_2} \min_j{dist(x_2,x_{1j}) } $$
where $dist()$ is the function we use as our vector distance metric. We tried an ensemble of distance functions but found that euclidean distance worked the best and was most intuitive to interpret the word2vec model. 

Another major consideration in our results was the evaluation of weighing each word2vec by parts of speech. what this means is that for some sentence we would extract the verbs and compute word mover distance for just that part of speech. We replicated this for three major classes: Noun, represented as the tag "NOUN". Proper Nouns with the tag "PROPN", and verbs with the tag "VERB". 
It was our hope that by introducing a variable for each part of speech that the differing weights would help improve our accuracy of regression. The resulting vecotrs\_\_

\begin{center}
\textbf{Cross Validation Results(with Log Loss function)}

	\begin{tabular}{|c|c|c|c|c|c|} \hline
	Method & Logistic & Linear & Ridge & LassoLars & SDG \\ \hline
	Custom WMD & \textbf{0.64659} & \textbf{0.64659} &\textbf{ 0.64659} & 0.65906 & 12.78577 \\ \hline
	unique words WMD & 0.67535 & \textbf{0.65143} & \textbf{0.65143} & 0.65906 & 12.78470 \\ \hline
	combined verbs and noun WMD & 0.67462 & \textbf{0.65875} & \textbf{0.65875} & 0.65906 & 12.78727 \\ \hline
	named entities WMD & 0.67568 & \textbf{0.65733} & \textbf{0.65733} & 0.65906 & 12.81439 \\ \hline
	Various parts of speech weighted & \textbf{0.65728} & 0.65766 & 0.65766 & 0.65906 & 17.49136\\ \hline
	Combined & \textbf{0.57243} & 0.75071 & 0.75070 & 0.65906 & 9.79779 \\ \hline
	\end{tabular}
\end{center}

\subsection{Ensemble Methods}

\section{Discussion and Conclusion}

Our methods were not competitive with the best submissions on Kaggle. As of this write-up, 86 teams had submissions with log loss under 0.2, and 577 under 0.3. However, these methods use \_\_neural networks\_\_. The disadvantage of neural networks is their difficulty to interpret. The methods that we used, especially the more successful ones, benefit from a high level of interpretability. 

For the word count method, the parts of speech with the highest coefficients were \_\_\_\_

BOW methods are particularly interpretable. The words that are the most predictive of duplicate questions are \_\_\_\_

Since \wordtvec\ embeds the words in a lower-dimensional space defined by a neural network, the embeddings themselves are not interpretable. However, when the WMDistances are calculated separately for different parts of speech, we can see which parts of speech need to be close, rather than exact matches.

Lastly, Wordnet \_\_\_\_

In future work, it will be interesting to\_\_\_\_

\subsubsection*{Acknowledgments}

We would like to thank Eric Mitchell and Bert Bertrand for their help and advice.

\printbibliography

\end{document}
